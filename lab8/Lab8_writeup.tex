% template created by: Russell Haering. arr. Joseph Crop
\documentclass[12pt,letterpaper]{article}
\usepackage{anysize}
\usepackage{courier}
\marginsize{2cm}{2cm}{1cm}{1cm}

\begin{document}

\begin{titlepage}
    \vspace*{4cm}
    \begin{flushright}
    {\huge
        ECE 375 Lab Final\\[1cm]
    }
    {\large
        Remote and Robot (USART)
    }
    \end{flushright}
    \begin{flushleft}
    Lab Time: Thursday 10am - 12
    \end{flushleft}
    \begin{flushright}
    Josh Matteson
    \vfill
    \rule{5in}{.5mm}\\
    TA Signature
    \end{flushright}

\end{titlepage}

\section{Introduction}
This lab is dedicated to understanding the USART that comes with the ATmega128 and its importance to embedded programming. We learned how to use the transmitter 
and receiver on the AVR board, and how it transmits packets and in what order. This along with how Bot ID's work. This involved setting up the USART1 beforehand,
which involved setting the baud rate, the packet type, and the transmittion type. We also initialized the external interrupts for Ports B and D, and setting up the 
interrupt masks. The importance of this lab is found in the everyday use of communication between different embedded systems, which occurs everywhere. 

\section{Program Overview}
{\noindent The overall point of this program is to get two avr boards to communicate with each other, one acting as a remote, and one acting as a robot. Someone 
will enter one of 6 commands on the remote, and the robot will respond with the appropriete command. This is represented by the LED's instead of actual wheels. 
We are also able to play a freeze tag game, where a robot receives a freeze command from another robot and has to freeze. We do this by sending a command on the 
remote which sends it to the board, which, in turn, sends it to all other boards around it.  \par}

\section{Robot.asm: Vectors}
We used three basic vectors for this program, the basic ones we used for the bump bot (0002 for HitRight, and 0004 for HitLeft), and 003C for our Receive routine 
(USART1).

\section{Robot.asm: Initialize Routine}
This initialize routine initializes the stack pointer, Initializes Port B for output, D for input. We then intialized the baud rate for USART1, and then configured 
the receive and transmit and enabled interrupts for receive. After that, we configured the packet to be 8 long with 2 stop bits. We set the interrupt mask and triggers 
for the bumpers just like the previous labs. Finally, we initialized some values for memory.

\section{Robot.asm: Main Program}
The main program is very simple, as it just simply loops through itself waiting to be interrupted.

\section{Robot.asm: USART Receive}
This function controls how we handle the USART1 interrupts. We first determine if it's an ID or action command but checking if the 7th bit is set, and then branching 
off into separate routines. We also check if we got frozen in this routine, and jump to FreezeRobot if we did.

\section{Robot.asm: CheckID}
CheckID compares the value that we received to the Bot ID that we have stored, and if they match, we give a value stored in memory a one, if not give it a zero.

\section{Robot.asm: CheckCommand}
This function test whether we had a correct Bot ID given to us or not, and then performs the command if we did. If we did not, we jump to the end of the routine and 
ignore the input.

\section{Robot.asm: FreezeRobot}
FreezeRobot means that our robot received a freeze command from another robot, and will freeze up and not respond to controls or interrupts for 5 seconds. We also 
decrement the counter in this function, checking if we have been frozen three times. 

\section{Robot.asm: FreezeOthers}
FreezeOthers is a function that gets called when the command sent to us by the remote is the freeze command. We send out a freeze signal to all other robots. Since 
we are sending this to everybody, our own robot gets hit, which means that we need to clear the receive buffer in order for our robot to not get frozen. 

\section{Robot.asm: USART Receiver end}
This is part of the USART Receive Function, and gets called when we finish the function. It pops all the variables and register off the stack and returns us to main. 

\section{Robot.asm: Die}
Die gets called when we have been frozen three times, and basically causes the robot to go into an infinite loop without a possibility of being interrupted. 

\section{Robot.asm: HitRight}
Handles the interupt received when hitting the right bumper on the tekbot. Causes the bot to move backwards for a second, turn left for a second, and then returns.

\section{Robot.asm: HitLeft}
Handles the interupt received when hitting the left bumper on the tekbot. Causes the bot to move backwards for a second, turn right for a second, and then returns.

\section{Remote.asm: Vectors}
No vectors were used for Remote.

\section{Remote.asm: Initialize Routine}
The Init routine is almost exactly the same as the robot Init routine, except we don't need interrupts or values set to anything. The Stack, i/o ports, and USART1 are 
the same.

\section{Remote.asm: Main}
The main program polls the inputs from PORTD, and goes into a specific subroutine depending on which button was pressed on the remote. 

\section{Remote.asm: XCommand}
All the subroutines that have the Command name in them perform a specific action for a specific button pushed. For example, The ForwardCommand first calls a function 
that sends the ID of our robot, and then sends the command to the robot. It loops the transmit finished flag from UCSR1A until we finish, and then returns back to main.

\section{Remote.asm: TransmitAddress}
TransmitAddress gets the address of the Bot, and then sends it to the robot before anything else. Just like the other commands, it polls until we've finished sending 
before continuing on with the rest of the program.

\section{Both: Wait}
The short function (used from the previous labs), causes the program to wait for 1 second before continuing.

\section{Difficulties}
By far the biggest difficulty was finding another AVR board to program with, after that, the rest of the lab was simply combining previous knowledge with some new challenges.

\section{Conclusion}
Overall, the importance of this lab was clearly seen and understood. Communication between multiple embedded systems is used all the time on lots of machinery. 
The more I learn and get an understanding of it now, the better I'll be able to understand them in the future and understand how to better program machinery 
in a futue job.

\section{Source Code for Robot.asm}
\scriptsize{\begin{verbatim}
;***********************************************************
;*
;*	Robot.asm
;*
;*	The program receives signals from the remote, and 
;* 	performs specific actions according to it.
;*
;*	This is the RECEIVE skeleton file for Lab 8 of ECE 375
;*
;***********************************************************
;*
;*	 Author: Josh Matteson
;*	   Date: 02/02/20017
;*
;***********************************************************

.include "m128def.inc"			; Include definition file

;***********************************************************
;*	Internal Register Definitions and Constants
;***********************************************************
.def	mpr = r16				; Multi-Purpose Register
.def 	mpr2 = r17				; Second Multi-Puporse Register
.def 	waitcnt = r18			; Wait counter
.def	ilcnt = r19				; Inner Loop Counter
.def	olcnt = r20				; Outer Loop Counter

.equ 	WTime = 100				; 1 second

.equ	WskrR = 0				; Right Whisker Input Bit
.equ	WskrL = 1				; Left Whisker Input Bit
.equ	EngEnR = 4				; Right Engine Enable Bit
.equ	EngEnL = 7				; Left Engine Enable Bit
.equ	EngDirR = 5				; Right Engine Direction Bit
.equ	EngDirL = 6				; Left Engine Direction Bit

.equ	BotAddress = 13 ;(Enter your robot's address here (8 bits))
.equ 	Command = $0100
.equ	FreezeCount = $0140

;/////////////////////////////////////////////////////////////
;These macros are the values to make the TekBot Move.
;/////////////////////////////////////////////////////////////
.equ	MovFwd =  (1<<EngDirR|1<<EngDirL)	;0b01100000 Move Forward Action Code
.equ	MovBck =  $00						;0b00000000 Move Backward Action Code
.equ	TurnR =   (1<<EngDirL)				;0b01000000 Turn Right Action Code
.equ	TurnL =   (1<<EngDirR)				;0b00100000 Turn Left Action Code
.equ	Halt = ($80|1<<(EngEnR-1)|1<<(EngEnL-1))	;0b10010000 Halt Action Code
.equ 	Freeze = 	0b01010101				;Command to freeze this robot
.equ 	SendFreeze = 0b11111000
.equ 	SendFreezeR = 0b011110000

;***********************************************************
;*	Start of Code Segment
;***********************************************************
.cseg							; Beginning of code segment

;***********************************************************
;*	Interrupt Vectors
;***********************************************************
.org	$0000					; Beginning of IVs
		rjmp 	INIT			; Reset interrupt

;Should have Interrupt vectors for:
;- Right whisker
.org 	$0002 ;{IRQ0 => pin0, PORTD}
		rcall HitRight ; Call hit right function
		reti ; Return from interrupt
;- Left whisker
.org 	$0004 ;{IRQ1 => pin1, PORTD}
		rcall HitLeft ; Call hit left function
		reti ; Return from interrupt
;- USART receive
.org 	$003C ; USART1, Rx complete interrupt
		rcall USART_Receive ; Call USART_Receive
		reti ; Return from interrupt

.org	$0046					; End of Interrupt Vectors

;***********************************************************
;*	Program Initialization
;***********************************************************
INIT:
	;Stack Pointer (VERY IMPORTANT!!!!)
	LDI 	mpr, LOW(RAMEND) ; Low Byte of End SRAM Address
	OUT 	SPL, mpr ; Write byte to SPL
	LDI 	mpr, HIGH(RAMEND) ; High Byte of End SRAM Address
	OUT 	SPH, mpr ; Write byte to SPH
	; I/O Ports
	; Initialize Port B for output
	ldi		mpr, $00		; Initialize Port B Data Register
	out		PORTB, mpr		; so all Port B outputs are low	
	ldi		mpr, $FF		; Set Port B Data Direction Register
	out		DDRB, mpr		; for output
	
	; Initialize Port D for input
	ldi		mpr, $FF		; Initialize Port D Data Register
	out		PORTD, mpr		; so all Port D inputs are Tri-State
	ldi		mpr, $00		; Set Port D Data Direction Register
	out		DDRD, mpr		; for input

	;USART1
	;Set baudrate at 2400bps
	; (16 MHz / (16 * 2400 [desired baud rate])) - 1 = 416
	ldi 	mpr, low(416) ; Load low byte into UBRR1L
	sts 	UBRR1L, mpr
	ldi 	mpr, high(416) ; Load high byte into UBRR1H
	sts 	UBRR1H, mpr
	;Don't know if this is necessary, but set UCSR1A to 0b00000000
	;ldi 	mpr, 0b00000000
	;sts 	UCSR1A, mpr
	;Enable receiver and enable receive interrupts, and transmit
	ldi 	mpr, (1<<RXEN1 | 1<<RXCIE1 | 1<<TXEN0) ; 
	sts		UCSR1B, mpr
	;Set frame format: 8 data bits, 2 stop bits
	ldi		mpr, (1<<USBS1 | 1<<UCSZ11 | 1<<UCSZ10 | 1<<UCSZ11) ; UCSZn1 and UCSZn0 to 
	; 1's for 8 data bits and USBSn set to 1 for 2 stopping bits
	sts 	UCSR1C, mpr
	;External Interrupts

	;Set the External Interrupt Mask
	ldi		mpr, 0b00000011 ; Enable INTO and INT1 for interrupts
	out		EIMSK, mpr
	;Set the Interrupt Sense Control to falling edge detection
	ldi		mpr, 0b00001010 ; Set INT0 and INT1 to trigger on falling edge
	sts		EICRA, mpr

	sei
	;Other
	ldi 	YL, low(Command)
	ldi 	YH, high(command)
	ldi		mpr, 0
	st		Y, mpr
	
	ldi 	ZL, low(FreezeCount)
	ldi 	ZH, high(FreezeCount)
	ldi		mpr, 0b00000011
	st		Z, mpr

;***********************************************************
;*	Main Program
;***********************************************************
MAIN:
	;TODO: ???
		rjmp	MAIN

;***********************************************************
;*	Functions and Subroutines
;***********************************************************

USART_Receive:

		push	mpr2
		push 	mpr
		in		mpr, SREG
		push	waitcnt			; Wait counter
		push 	mpr
		push	YL
		push	YH
		push	ZL
		push	ZH

		lds		mpr, UDR1
		cpi		mpr, Freeze
		breq	FreezeRobot
		sbrc	mpr, 7
		rjmp	CheckCommand
		rjmp	CheckID

CheckID:

		ldi 	YL, low(Command)
		ldi 	YH, high(command)
		ld		mpr2, Y
		
		cpi 	mpr, BotAddress
		breq	Correct
		ldi		mpr2, 0
		st		Y, mpr2

		rjmp 	USART_Receiver_end

Correct:

		ldi		mpr2, 1
		st		Y, mpr2

		rjmp 	USART_Receiver_end

CheckCommand:

		ldi 	YL, low(Command)
		ldi 	YH, high(command)
		ld		mpr2, Y
		sbrs	mpr2, 0
		rjmp	USART_Receiver_end
		rol 	mpr
		;ldi		waitcnt, WTime	; Wait for 1 second
		;rcall	Wait
		cpi		mpr, SendFreezeR
		breq	FreezeOthers
		out		PORTB, mpr

		rjmp	USART_Receiver_end

FreezeRobot:

		; Turn off interupts for duration of function
		cli

		ldi		mpr2, Halt
		rol		mpr2
		out		PORTB, mpr2
		ldi		waitcnt, WTime	; Wait for 1 second
		rcall	Wait

		ldi		ZL, low(FreezeCount)
		ldi		ZH, high(FreezeCount)
		ld		mpr, Z

		dec		mpr
		breq	Die
		st		Z, mpr

		; Turn interupts back on
		ldi		mpr, 0b00000011
		out 	EIFR, mpr ; Clear External Interupt Flag Register
		sei

		rjmp 	USART_Receiver_end

FreezeOthers:

		ldi		mpr, 0b01010101
		out		PORTB, mpr
		sts		UDR1, mpr
Send_Loop:
		lds		mpr, UCSR1A
		sbrs	mpr, TXC1
		rjmp	Send_Loop
Clear_Input:
		lds 	mpr, UCSR1A
		sbrs 	mpr, RXC1
		rjmp 	Clear_Input
		lds		mpr, UDR1

		rjmp	USART_Receiver_end

		
USART_Receiver_end:

		pop		ZH
		pop 	ZL
		pop 	YH
		pop		YL
		pop 	mpr
		pop		waitcnt
		out		SREG, mpr
		pop 	mpr
		pop 	mpr2
		ret

Die:
		rjmp 	Die


HitRight:							; Begin a function with a label

		push	mpr			; Save mpr register
		push	waitcnt		; Save wait register
		in		mpr, SREG	; Save program state
		push	mpr			;

		; Turn off interupts for duration of function
		cli

		; Move Backwards for a second
		ldi		mpr, MovBck	; Load Move Backward command
		out		PORTB, mpr	; Send command to port
		ldi		waitcnt, WTime	; Wait for 1 second
		rcall	Wait			; Call wait function

		; Turn left for a second
		ldi		mpr, TurnL	; Load Turn Left Command
		out		PORTB, mpr	; Send command to port
		ldi		waitcnt, WTime	; Wait for 1 second
		rcall	Wait			; Call wait function

		; Move Forward again	
		ldi		mpr, MovFwd	; Load Move Forward command
		out		PORTB, mpr	; Send command to port

		; Turn interupts back on
		ldi		mpr, 0b00000011
		out 	EIFR, mpr ; Clear External Interupt Flag Register
		sei

		pop		mpr		; Restore program state
		out		SREG, mpr	;
		pop		waitcnt		; Restore wait register
		pop		mpr		; Restore mpr
		
		ret						; End a function with RET

HitLeft:							; Begin a function with a label

		push	mpr			; Save mpr register
		push	waitcnt		; Save wait register
		in		mpr, SREG	; Save program state
		push	mpr			;

		; Turn off interupts for duration of function
		cli

		; Move Backwards for a second
		ldi		mpr, MovBck	; Load Move Backward command
		out		PORTB, mpr	; Send command to port
		ldi		waitcnt, WTime	; Wait for a second
		rcall	Wait			; Call wait function

		; Turn right for a second
		ldi		mpr, TurnR	; Load Turn Left Command
		out		PORTB, mpr	; Send command to port
		ldi		waitcnt, WTime	; Wait for a second
		rcall	Wait			; Call wait function

		; Move Forward again	
		ldi		mpr, MovFwd	; Load Move Forward command
		out		PORTB, mpr	; Send command to port

		; Turn interupts back on
		ldi		mpr, 0b00000011 ; Left and Right bumper can interrupt again
		out 	EIFR, mpr ; Clear External Interupt Flag Register
		sei

		pop		mpr		; Restore program state
		out		SREG, mpr	;
		pop		waitcnt		; Restore wait register
		pop		mpr		; Restore mpr

		ret				; Return from subroutine

Wait:
		push	waitcnt			; Save wait register
		push	ilcnt			; Save ilcnt register
		push	olcnt			; Save olcnt register
; Wait function from previous labs
Loop:	
		ldi		olcnt, 224		; load olcnt register
OLoop:	
		ldi		ilcnt, 237		; load ilcnt register
ILoop:	
		dec		ilcnt			; decrement ilcnt
		brne	ILoop			; Continue Inner Loop
		dec		olcnt		; decrement olcnt
		brne	OLoop			; Continue Outer Loop
		dec		waitcnt		; Decrement wait 
		brne	Loop			; Continue Wait loop	

		pop		olcnt		; Restore olcnt register
		pop		ilcnt		; Restore ilcnt register
		pop		waitcnt		; Restore wait register
		ret					; Return from subroutine

;***********************************************************
;*	Stored Program Data
;***********************************************************

;***********************************************************
;*	Additional Program Includes
;***********************************************************


\end{verbatim}}

\section{Source Code for Robot.asm}
\scriptsize{\begin{verbatim}
;***********************************************************
;*
;*	Remote.asm
;*
;*	Transmits signals to the robot giving it commands.
;*
;*	This is the TRANSMIT skeleton file for Lab 8 of ECE 375
;*
;***********************************************************
;*
;*	 Author: Josh Matteson
;*	   Date: 02/02/20017
;*
;***********************************************************

.include "m128def.inc"			; Include definition file

;***********************************************************
;*	Internal Register Definitions and Constants
;***********************************************************
.def	mpr = r16				; Multi-Purpose Register
.def 	waitcnt = r18			; Wait counter
.def	ilcnt = r19				; Inner Loop Counter
.def	olcnt = r20				; Outer Loop Counter

.equ	EngEnR = 4				; Right Engine Enable Bit
.equ	EngEnL = 7				; Left Engine Enable Bit
.equ	EngDirR = 5				; Right Engine Direction Bit
.equ	EngDirL = 6				; Left Engine Direction Bit

.equ 	WTime = 25				; 1 second

; Use these action codes between the remote and robot
; MSB = 1 thus:
; control signals are shifted right by one and ORed with 0b10000000 = $80
.equ	MovFwd =  ($80|1<<(EngDirR-1)|1<<(EngDirL-1))	;0b10110000 Move Forward Action Code
.equ	MovBck =  ($80|$00)								;0b10000000 Move Backward Action Code
.equ	TurnR =   ($80|1<<(EngDirL-1))					;0b10100000 Turn Right Action Code
.equ	TurnL =   ($80|1<<(EngDirR-1))					;0b10010000 Turn Left Action Code
.equ	Halt =    ($80|1<<(EngEnR-1)|1<<(EngEnL-1))		;0b11001000 Halt Action Code
.equ	Freeze =  0b11111000

.equ	SendComplete = 0b01100000

.equ	BotAddress = 13 ;(Enter your robot's address here (8 bits))
; 0b00001101

;***********************************************************
;*	Start of Code Segment
;***********************************************************
.cseg							; Beginning of code segment

;***********************************************************
;*	Interrupt Vectors
;***********************************************************
.org	$0000					; Beginning of IVs
		rjmp 	INIT			; Reset interrupt

.org	$0046					; End of Interrupt Vectors

;***********************************************************
;*	Program Initialization
;***********************************************************
INIT:
	;Stack Pointer (VERY IMPORTANT!!!!)
	LDI 	mpr, LOW(RAMEND) ; Low Byte of End SRAM Address
	OUT 	SPL, mpr ; Write byte to SPL
	LDI 	mpr, HIGH(RAMEND) ; High Byte of End SRAM Address
	OUT 	SPH, mpr ; Write byte to SPH
	;I/O Ports
	; Initialize Port B for output
	ldi		mpr, $00		; Initialize Port B Data Register
	out		PORTB, mpr		; so all Port B outputs are low	
	ldi		mpr, $FF		; Set Port B Data Direction Register
	out		DDRB, mpr		; for output
	
	; Initialize Port D for input
	ldi		mpr, $FF		; Initialize Port D Data Register
	out		PORTD, mpr		; so all Port D inputs are Tri-State
	ldi		mpr, $00		; Set Port D Data Direction Register
	out		DDRD, mpr		; for input
	;USART1
	;Set baudrate at 2400bps
	; (16 MHz / (16 * 2400 [desired baud rate])) - 1 = 416
	ldi 	mpr, low(416) ; Load low byte into UBRR1L
	sts 	UBRR1L, mpr 
	ldi 	mpr, high(416) ; Load high byte into UBRR1H
	sts 	UBRR1H, mpr

	;Enable transmitter
	ldi		mpr, (1<<TXEN0)
	sts		UCSR1B, mpr
	;Set frame format: 8 data bits, 2 stop bits
	ldi		mpr, (1<<USBS1 | 1<<UCSZ11 | 1<<UCSZ10 | 1<<UPM11) ; UCSZn1 and UCSZn0 to 
	; 1's for 8 data bits and USBSn set to 1 for 2 stopping bits
	sts 	UCSR1C, mpr

	sei

	;Other

;***********************************************************
;*	Main Program
;***********************************************************
MAIN:
		in		mpr, PIND		; Get button input from Port D
		cpi		mpr, 0b11111110	; Check for first button input
		brne	NEXT1			; Continue with next check
		rcall	ForwardCommand		; Call the subroutine ForwardCommand
		rjmp	MAIN			; Loop back to main
NEXT1:	
		cpi		mpr, 0b11111101	; Check for 2nd button input 
		brne	NEXT2			; No button input, continue program
		rcall	BackwardCommand			; Call subroutine BackwardCommand
		rjmp	MAIN			; Continue through main
NEXT2:	
		cpi		mpr, 0b11101111	; Check for 5th button input 
		brne	NEXT3			; No button input, continue program
		rcall	TurnRCommand			; Call subroutine TurnRCommand
		rjmp	MAIN			; Continue through main
NEXT3:	
		cpi		mpr, 0b11011111	; Check for 6th button input 
		brne	NEXT4			; No button input, continue program
		rcall	TurnLCommand			; Call subroutine TurnLCommand
		rjmp	MAIN			; Continue through main
NEXT4:	
		cpi		mpr, 0b10111111	; Check for 7th button input 
		brne	NEXT5			; No button input, continue program
		rcall	HaltCommand		; Call subroutine HaltCommand
		rjmp	MAIN			; Continue through main
NEXT5:	
		cpi		mpr, 0b01111111	; Check for 8th button input 
		brne	MAIN			; No button input, continue program
		rcall	FreezeCommand			; Call subroutine FreezeCommand
		rjmp	MAIN			; Continue through main

;***********************************************************
;*	Functions and Subroutines
;***********************************************************
ForwardCommand:

		rcall TransmitAddress
		
		ldi		mpr, MovFwd
		out		PORTB, mpr
		sts		UDR1, mpr

		
Forward_Loop:
		lds		mpr, UCSR1A
		cpi		mpr, SendComplete
		brne	Forward_Loop

		ret

BackwardCommand:

		rcall TransmitAddress

		ldi		mpr, MovBck
		out		PORTB, mpr
		sts		UDR1, mpr

Backward_Loop:
		lds		mpr, UCSR1A
		cpi		mpr, SendComplete
		brne	Backward_Loop

		ret

TurnRCommand:

		rcall TransmitAddress
		
		ldi		mpr, TurnR
		out		PORTB, mpr
		sts		UDR1, mpr

TurnR_Loop:
		lds		mpr, UCSR1A
		cpi		mpr, SendComplete
		brne	TurnR_Loop

		ret

TurnLCommand:

		rcall TransmitAddress
		
		ldi		mpr, TurnL
		out		PORTB, mpr
		sts		UDR1, mpr

TurnL_Loop:
		lds		mpr, UCSR1A
		cpi		mpr, SendComplete
		brne	TurnL_Loop

		ret

HaltCommand:

		rcall TransmitAddress
		
		ldi		mpr, Halt
		out		PORTB, mpr
		sts		UDR1, mpr

Halt_Loop:
		lds		mpr, UCSR1A
		cpi		mpr, SendComplete
		brne	Halt_Loop

		ret

;FreezeCommand:

		;ldi		mpr, 0b01010101
		;out		PORTB, mpr
		;sts		UDR1, mpr

;Freeze_Loop:
		;lds		mpr, UCSR1A
		;sbrs	mpr, TXC1
		;rjmp	Freeze_Loop

		;ldi		waitcnt, WTime	; Wait for 1 second
		;rcall	Wait

		;ret

FreezeCommand:

		rcall TransmitAddress
		
		ldi		mpr, Freeze
		out		PORTB, mpr
		sts		UDR1, mpr

Freeze_Loop:
		lds		mpr, UCSR1A
		cpi		mpr, SendComplete
		brne	Freeze_Loop

		ldi		waitcnt, WTime	; Wait for 1 second
		rcall	Wait

		ret

TransmitAddress:

		ldi		mpr, BotAddress
		sts		UDR1, mpr

Transmit_Loop:
		lds		mpr, UCSR1A
		cpi		mpr, SendComplete
		brne	Transmit_Loop

		ret
; Wait function from previous labs
Wait:
		push	waitcnt			; Save wait register
		push	ilcnt			; Save ilcnt register
		push	olcnt			; Save olcnt register

Loop:	
		ldi		olcnt, 224		; load olcnt register
OLoop:	
		ldi		ilcnt, 237		; load ilcnt register
ILoop:	
		dec		ilcnt			; decrement ilcnt
		brne	ILoop			; Continue Inner Loop
		dec		olcnt		; decrement olcnt
		brne	OLoop			; Continue Outer Loop
		dec		waitcnt		; Decrement wait 
		brne	Loop			; Continue Wait loop	

		pop		olcnt		; Restore olcnt register
		pop		ilcnt		; Restore ilcnt register
		pop		waitcnt		; Restore wait register
		ret					; Return from subroutine

;***********************************************************
;*	Stored Program Data
;***********************************************************

;***********************************************************
;*	Additional Program Includes
;***********************************************************
\end{verbatim}}
\end{document}
